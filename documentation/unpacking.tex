
\chapter{Unpacking data from TRAGALDABAS}

\section{DST\_Guide}

En el directorio \texttt{/media/Datos2TB/username/tragaldabas/soft\_TT/} tenemos los siguientes archivos, que se pueden ejecutar por este orden:

\begin{itemize}
	\item \textit{createFileLists.py}: Genera las listas con paths absolutos los archivos \textit{tryydddhhmmss.hld} tanto del \texttt{/media/Datos2TB/tragaldabas/data/done} como del \texttt{/media/externalHD/Tragaldabas/data\_hld} y los guarda en archivos \textit{list\_yyyy\_day\_ddd.list}, uno por cada día de archivos .hld. Además crea una carpeta con ellos y unas listas \textit{listExternal.list} con los nombres de los ficheros que ha encontrado en cada disco duro.
	
	\item \textit{efficiency.cc}: La ayuda para compilar este código se encuentra en \textit{howtocompile.txt}
	
	\begin{lstlisting}[style=customsh]
[username@fptrucha soft_TT]$ g++ -Wall -g -O -fPIC 'root-config --cflags' <filename.cc> 'root-config --glibs' 'root-config --libs' -L ./ -I ./ -ltunpacker -o <analysis_exe>
	\end{lstlisting}
	donde \textless filename.cc\textgreater\ es en este caso \textit{efficiency.cc} y en \textless analysis\_exe\textgreater\ se le da el nombre que se quiera al archivo ejecutable que crea con los parámetros del detector.
	
	Antes de proceder a la compilación, definir manualmente dentro de \textit{efficiency.cc} (o descomentar, si ya están escritas) solamente las siguientes líneas:
	
	\begin{lstlisting}[style=customc]
void doStaffCalibration(char* path,char* name){
	unpack.setFileLookupPar(<path_to_luptab>)
	// Escoger la luptab correspondiente a las fechas para las cuales se estan analizando estos datos concretos.
	unpack.setFileHitFinderPar(`empty_cal_2016.txt>`)
	// Estos datos son siempre los mismos.
	unpack.setFileHitFinderParOut(`<my_path>/20YYDST/pars/` + day + `_CalPars.txt`)
	// Aqui solamente introducir el path hacia tu directorio soft_TT.
	unpack.fillCalibration(`/media/Datos2TB/tragaldabas/data/done/`, `../` + fName, `../qcalhistos2/`, day + `\_qcalhistos.root`, 1000000, 0)
	// En esta linea, introducir correctamente los paths.
	}
	
void doStaffAnalysis(char* path,char* name){
	fill.setFileLookupPar(`luptabs/luptab.txt`)
	// Editar el path hacia la luptab correspondiente.
	fill.setFileHitFinderPath(`/day\_CalPars.txt`)
	// El output de unpack.setFileHitFinderParOut() sera el input de este metodo: introducir el path hacia el mismo.
	}
	
	/* 
	[...]
	*/
	\end{lstlisting}

	\item \textit{multiThreadRun.py}: Multiprocesado en varios cores para agilizar el proceso. Se lanza desde bash de la siguiente forma
	
	\begin{lstlisting}[style=customsh]
[username@fptrucha soft_TT]$ python3 multithreadRun.py ./analysis_exe lists_name.txt
	\end{lstlisting}
	siendo el primer argumento el nombre del archivo ejecutable devuelto por \textit{efficiency.cc} y el segundo el archivo con la lista de nombres devuelto por \textit{createFileLists.py}.
	
	\textbf{20XXDST/} new\_results.* para ver el root y en cuentas de celdas un valor de 2500 es razonable como máximo.
\end{itemize}

Al lanzar el ejecutable anterior, se generan archivos en los directorios \texttt{qcalhistos}\footnote{Histogramas de calibración} y \texttt{pars}, los cuales hay que crear manualmente junto a \texttt{../20YYDST/results}, donde \texttt{20YY} es el año.

``Matar celdas que hayan registrado ruído y volver a correr sobre los .hld''

\subsection{Proceso:}

(¿?) Primero creamos el archivo vacío de celdas activas con:
\begin{lstlisting}[style=customsh]
[username@fptrucha soft_TT]$ $ root -l CreateActiveCells.c
\end{lstlisting}
y a continuación
\begin{lstlisting}[style=customsh]
[username@fptrucha soft_TT]$ $ make clean
[username@fptrucha soft_TT]$ $ make
\end{lstlisting}

Blah blah

...

Crear directorio \texttt{20YYDST} manualmente en \texttt{/media/Datos2TB/username/} como se muestra en la figura \ref{fg:20YYDSTdir}

\begin{figure}[!h]
\begin{forest}
  pic dir tree,
  pic root,
  for tree={directory,},
	[/media/Datos2TB/username/
		[20YYDST/
			[qcalhistos/]
			[results/]
			[pars/]
		]
		[...]
		[..., file]
	]
\end{forest}
\label{fg:20YYDSTdir}
\caption{Directorio \texttt{20YYDST/} y subdirectorios.}
\end{figure}

A continuación, ejecutar los programas anteriormente mencionados. Notar que se pueden usar las flags:
\begin{itemize}
	\item \texttt{htop}: comando para ver procesos
	\item \texttt{root -l GoodActiveCells.root}: para abrir con \texttt{new TBrowser} más fácilmente (se queda arriba de todo en el árbol). Con esto se pueden ver las celdas muertas
	\item \texttt{hadd -f dst2018\_full.root *.root}: Une varios histogramas root (*.root) en un solo archivo (dst2038\_full.root).
\end{itemize}


Crear archivo vacio sin abrirlo
\begin{lstlisting}[style=customsh]
[username@fptrucha soft_TT]$ touch luptab.txt
[username@fptrucha soft_TT]$ ln -sf luptab\_20201013.txt luptab.txt
\end{lstlisting}
y crear un hard link\footnote{También llamado enlace simbólico.}. A continuación abrir \textit{efficiency.cc} y cambiar paths.

Exportar al load library path
\begin{lstlisting}[style=customsh]
[username@fptrucha any_dir]$ export LD_LIBRARY_PATH=$LD_LIBRARY_PATH:<path/to/libtunpacker.so>
\end{lstlisting}
donde \texttt{<path/to/libtunpacker.so>} es el path absoluto al direcotrio \texttt{/media/.../ username/tragaldabas/soft\_TT/}

