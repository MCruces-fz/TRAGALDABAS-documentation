\chapter{Sorting Data}

Una de las primeras conclusiones es que el sistema de referencia para datos reales está localizado con Z=0 en el suelo, pero Z positivas hacia el techo, y la relación del número de TRB (y por consiguiente la Z) con el plano la puedes ver en los scripts de monitorización:

\begin{lstlisting} % [style=customc]
can->cd(1);
T->Draw("rpcraw.fRow-1:rpcraw.fCol-1>>T1(12,0,12,10,0,10)",
        "rpcraw.fTrbnum==2","colz text");
T1->SetTitle("XXXXXXXXX;column;row");can_1->SetLogz();
T1->GetZaxis()->SetRangeUser(0,10000);
can->cd(2);
T->Draw("rpcraw.fRow-1:rpcraw.fCol-1>>T3(12,0,12,10,0,10)",
        "rpcraw.fTrbnum==0","colz text");
T3->SetTitle("XXXXXXXXX;column;row");can_2->SetLogz();
T3->GetZaxis()->SetRangeUser(0,10000);
can->cd(3);
T->Draw("rpcraw.fRow-1:rpcraw.fCol-1>>T4(12,0,12,10,0,10)",
        "rpcraw.fTrbnum==1","colz text");
T4->SetTitle("XXXXXXXXX;column;row");
can_3->SetLogz();
T4->GetZaxis()->SetRangeUser(0,10000);
\end{lstlisting}

Resumiendo:

$T1 = TRB 2 = Z \sim 1800$ mm

$T3 = TRB 0 = Z \sim  950$ mm

$T4 = TRB 1 = Z \sim   87$ mm

Respecto a lo de la carga, repasa con Damián (en copia) como están las cosas a nivel de rpchit. En el hitfinder se le resta el pedestal y luego se convierte a unidades. No se si ese pedestal que se le resta es el espacio en el que no hay nada (para poner el espectro de carga a empezar a cero) o también el corte después de ajustar el pico de disparos de carga cero, a partir del cual se consideran cargas válidas. Si es esto último, habría que repasar esa calibración ya que, como vimos por la mañana, se ve ese pico en los datos.

No tenemos nivel "cal" (calibrado), pasamos directamente de "raw" a "hit

---------


