\documentclass[a4paper]{article}
%\usepackage[subpreambles=true]{standalone}
\usepackage[utf8]{inputenc}
%\usepackage[T1]{fontenc}
\usepackage[english]{babel}

\usepackage{graphicx}
\usepackage{booktabs}
\usepackage{amsmath}
\usepackage{listings}
\usepackage{xcolor}
%\lstset { %
%%    language=C++,
%    backgroundcolor=\color{black!10}, % set backgroundcolor
%%    breaklines=true,
%%    frame=tlbr,
%    xleftmargin=0.5cm,
%    basicstyle=\ttfamily,
%    basicstyle=\footnotesize,% basic font setting
%}

\lstdefinestyle{customc}{
  belowcaptionskip=1\baselineskip,
  breaklines=true,
  frame=L,
  xleftmargin=\parindent,
  language=C++,
  morekeywords={*, Int_t, UInt_t, Float_t, Bool_t, TMatrixF, TString, ClonesArray, TClonesArray}, 
  numbers=left,
  showstringspaces=false,
  basicstyle=\footnotesize\ttfamily,
  keywordstyle=\bfseries\color{green!40!black},
  commentstyle=\itshape\color{purple!40!black},
  backgroundcolor=\color{yellow!10},
  identifierstyle=\color{black},
  stringstyle=\color{orange}\itfamily,
}

\lstdefinestyle{customsh}{
  language=bash,
  basicstyle=\color{black!80}\small\ttfamily,
  numbers=left,
  numberstyle=\color{black}\small\ttfamily,
  numbersep=3pt,
  % frame=tb,
  showstringspaces=false,
  breaklines=true,
  columns=fullflexible,
  backgroundcolor=\color{black!5},
  linewidth=0.9\linewidth,
  xleftmargin=\parindent,  % 0.05\linewidth,
  xrightmargin=-0.1\linewidth
}


\usepackage{hyperref}
\usepackage{breakurl}

\let\vec\mathbf  % Bold vectors

\graphicspath{{images/}}

\usepackage[edges]{forest}
\definecolor{folderbg}{RGB}{124,166,198}
\definecolor{folderborder}{RGB}{110,144,169}
\newlength\Size
\setlength\Size{4pt}
\tikzset{%
  folder/.pic={%
    \filldraw [draw=folderborder, top color=folderbg!50, bottom color=folderbg] (-1.05*\Size,0.2\Size+5pt) rectangle ++(.75*\Size,-0.2\Size-5pt);
    \filldraw [draw=folderborder, top color=folderbg!50, bottom color=folderbg] (-1.15*\Size,-\Size) rectangle (1.15*\Size,\Size);
  },
  file/.pic={%
    \filldraw [draw=folderborder, top color=folderbg!5, bottom color=folderbg!10] (-\Size,.4*\Size+5pt) coordinate (a) |- (\Size,-1.2*\Size) coordinate (b) -- ++(0,1.6*\Size) coordinate (c) -- ++(-5pt,5pt) coordinate (d) -- cycle (d) |- (c) ;
  },
}
\forestset{%
  declare autowrapped toks={pic me}{},
  declare boolean register={pic root},
  pic root=0,
  pic dir tree/.style={%
    for tree={%
      folder,
      font=\ttfamily,
      grow'=0,
    },
    before typesetting nodes={%
      for tree={%
        edge label+/.option={pic me},
      },
      if pic root={
        tikz+={
          \pic at ([xshift=\Size].west) {folder};
        },
        align={l}
      }{},
    },
  },
  pic me set/.code n args=2{%
    \forestset{%
      #1/.style={%
        inner xsep=2\Size,
        pic me={pic {#2}},
      }
    }
  },
  pic me set={directory}{folder},
  pic me set={file}{file},
}

\title{Directory distribution in Datos4TB \\External HDD}
\date{}

\begin{document}
\maketitle


To check those directories from \texttt{fptrucha.usc.es}, you can use the \texttt{tree} command
\begin{lstlisting}[style=customsh]
$ cd /media/Datos4TB/
$ tree -d -L 2 -C
\end{lstlisting}
where the flag \texttt{-d} shows only directories; \texttt{-L 2} to descend only two directories deep; \texttt{-C} turns colorization on. For more information about \texttt{tree} flags and options, see
\begin{lstlisting}[style=customsh]
$ tree --help
\end{lstlisting}


\section{Top directories}

\begin{forest}
  pic dir tree,
  pic root,
  for tree={directory,},
	[/media/Datos4TB/
        [cosmix/
            [context/]
            [data/]
            [pool/]
            [soft/]
        ]
		[people/
            [korna/]
            [mabbrescia/]
            [marwan/]
            [mcruces/]
        ]
		[tragaldabas/
            [context/]
            [data/]
            [people/]
            [pool/]
            [soft/]
		]
		[tristan/
            [context/]
            [data/]
            [DATOS\_TRISTAN\_Base/]
            [pool/]
            [soft/]
		]
		[...]
	]
\end{forest}

\section{tragaldabas/}

\begin{forest}
  pic dir tree,
  pic root,
  for tree={directory,},
	[/media/Datos4TB/tragaldabas/
		[context/
            [meteo/]
        ]
		[data/
            [dst/]
            [hlds/]
            [monitor/
                [cellmaps/]
                [png/]
            ]
            [Nov2020\_dst/
                [dat/]
                [png/]
                [root/]
            ]
            [slow/]
        ]
        [people/]
        [pool/]
        [soft/
            [calibrate/]
            [scripts/]
            [tracking/]
        ]
    ]
\end{forest}


\section{tristan/}

\begin{forest}
  pic dir tree,
  pic root,
  for tree={directory,},
	[/media/Datos4TB/tristan/
		[context/
            [meteo/]
        ]
		[data/
            [backup\_Jan20/]
            [dst/]
            [hlds/]
            [monitor/
                [cellmaps/]
                [png/]
            ]
            [raw/]
            [slow/]
            [zold\_backup/]
        ]
        [DATOS\_TRISTAN\_Base/
            [hlds/]
            [sc/]
        ]
        [pool/]
        [soft/
            [calibrate/]
            [scripts/]
            [tracking/]
        ]
    ]
\end{forest}


\end{document}
